%\pagestyle{fancy}
\chapter{CONCLUSIONES}
\begin{enumerate}
	\item La Municipalidad Distrital de Tamburco se ha enfocado en la ejecución de IOAAR \acrlong{ioarr} y mantenimientos. Esto se debe a las limitaciones de recursos, ya que la realización de estudios para proyectos de inversión pública (\acrshort{pip}) requeriría contar con oficinas más amplias y mantener un plantel de profesionales, lo cual implica asignar recursos adicionales. Esta estrategia permite optimizar los recursos disponibles y concentrarse en la ejecución de obras y acciones que beneficien directamente a la comunidad, aunque implique limitar la capacidad de realizar estudios detallados para nuevos proyectos de inversión pública.

	\item Es importante tener en cuenta que los practicantes que opten por realizar sus prácticas pre-profesionales en la Unidad de Estudios y Proyectos deben contar con conocimientos en diversos software de ingeniería. Estos incluyen herramientas como ETABS, SAFE y SAP2000, que son esenciales para el análisis y diseño de estructuras. Asimismo, se requiere experiencia en programas de dibujo asistido por computadora y modelado, como AutoCAD, Civil 3D y Revit, que son fundamentales para la elaboración de planos y modelos tridimensionales.
	Estos conocimientos permitirán a los practicantes contribuir de manera efectiva en la Unidad de Estudios y Proyectos, participando en la creación y desarrollo de proyectos de ingeniería de manera precisa y eficiente.

	\item Además de los conocimientos en software de ingeniería estructural y diseño, es igualmente importante que los practicantes tengan habilidades en áreas como Costos y Presupuestos, así como en Control y Seguimiento de Proyectos. En estas disciplinas, existen una amplia gama de software disponibles y el practicante puede elegir aquel con el que se sienta más cómodo o tenga experiencia previa. Esto asegura que puedan contribuir de manera efectiva en la planificación, gestión y control de los recursos y actividades relacionadas con los proyectos de ingeniería. La versatilidad en el manejo de diferentes herramientas es una habilidad valiosa para un profesional en el campo de la ingeniería.

	\item Es fundamental destacar que todos los criterios y decisiones tomadas en el proceso se basan rigurosamente en las normativas tanto nacionales como internacionales. Esto implica la lectura y comprensión detallada de reglamentos, manuales y demás documentos normativos pertinentes. Esta estricta adherencia a las regulaciones garantiza que los proyectos y procesos se desarrollen con los más altos estándares de calidad y seguridad, además de asegurar la legalidad y cumplimiento normativo en todas las etapas del trabajo.

	\item  Al concluir las prácticas pre-profesionales, el estudiante no solo adquiere conocimientos teóricos, sino también experiencias prácticas y aprendizajes invaluablemente enriquecedores al interactuar con profesionales experimentados en situaciones específicas y particulares. Cada caso y proyecto presenta desafíos y particularidades únicas, lo que permite al estudiante obtener una comprensión más profunda y aplicada de su futura profesión. Esta experiencia práctica es fundamental para su desarrollo profesional y le brinda una visión más completa y realista del campo en el que se desenvolverá.

	\item Al finalizar las prácticas pre-profesionales, el practicante no solo adquiere experiencia laboral, sino también establece relaciones laborales y amistades en el entorno profesional. Estas conexiones pueden jugar un papel crucial en el inicio de la carrera profesional, ya sea a través de recomendaciones para futuros trabajos remunerados o incluso mediante la posibilidad de ser contratado directamente por las personas con las que se trabajó durante las prácticas. El clima laboral positivo y las relaciones profesionales sólidas que se construyen durante esta etapa son activos valiosos que pueden abrir puertas importantes en el camino hacia una carrera exitosa.
\end{enumerate}