%\pagestyle{headings}
\chapter{RECOMENDACIONES}
	\begin{enumerate}
		
		\item Se recomienda a la entidad, establecer directrices claras y uniformes para la entrega y recepción de trabajos es fundamental para asegurar la calidad y consistencia en los proyectos realizados. Esto incluye especificar los componentes que deben estar incluidos en cada entrega, los formatos aceptables, los plazos y cualquier otra información relevante. Esta estandarización facilita el proceso de evaluación y asegura que todas las partes involucradas estén al tanto de lo que se espera en cada entrega. Así, se garantiza un proceso más eficiente y una mayor calidad en los trabajos realizados.
		
		\item Se recomienda a los practicantes contar con habilidades en el manejo de software especializado en ingeniería civil es esencial para contribuir de manera efectiva en proyectos y prácticas profesionales. Esto incluye herramientas como AutoCAD, Civil 3D, Revit, ETABS, SAFE, S10, MS Project, entre otros. Estos programas son fundamentales en la industria y permiten a los ingenieros realizar tareas de diseño, análisis y gestión de proyectos de manera eficiente. Quienes poseen habilidades en estas herramientas tienen un valor significativamente mayor en el campo laboral y están mejor preparados para aportar al desarrollo de proyectos de ingeniería civil.
		
		\item Para tomar decisiones efectivas en el entorno laboral, es crucial evitar depender exclusivamente de conocimientos empíricos, los cuales pueden ser inadecuados en muchas ocasiones. En su lugar, se recomienda recurrir a fuentes confiables como el jefe inmediato y otros expertos en el campo, buscando alinear las decisiones con las normativas vigentes. Esta consulta proporciona valiosos insights y perspectivas diversas, permitiendo llegar a soluciones más sólidas. Asimismo, es esencial mantenerse al tanto de las regulaciones actuales para garantizar la legalidad, seguridad y calidad en las acciones emprendidas.
		\item Durante mi experiencia como practicante, pude constatar un factor crucial en el ámbito de la remodelación y mantenimiento de espacios: la falta de consideración de las especialidades de comunicaciones, como el cableado de televisión e internet, ha sido una fuente recurrente de problemas. En el contexto actual, donde los servicios de internet, intercomunicadores y televisión son cada vez más relevantes en nuestra vida cotidiana, es imperativo reconocer la necesidad de integrar adecuadamente estas tecnologías en los proyectos de construcción y renovación.
		
	\end{enumerate}