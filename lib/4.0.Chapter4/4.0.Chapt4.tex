%\pagestyle{fancy}
\chapter{APRECIACIÓN GENERAL}
Durante mi período como practicante, pude obtener percepciones valiosas en tres niveles distintos: en la entidad en sí, en la \acrshort{uep} (Unidad de Estudios y Proyectos), y, por último, en el entorno laboral en general. Estas experiencias me permitieron adquirir un entendimiento más completo y enriquecedor de la dinámica y funcionamiento de cada uno de estos ámbitos.
\section{Entidad}
\begin{enumerate}
	\item La entidad denominado \acrlong{mdt} está representado por Raúl Silva Campos como el alcalde del mismo, según la figura \ref{fig:organigrama-tamburco}, las decisiones económicas se toma en consejo municipal según la norma denominada Ley Orgánica de Municipalidades, aprobado por  \cite{CongresoRepublica2003}.
	\item Es evidente que en este entorno laboral, las decisiones presupuestarias se abordan con un enfoque responsable y cuidadoso. Al ser un proceso llevado a cabo en consejo, se busca asegurar la máxima responsabilidad en la toma de decisiones. Esta consideración colectiva demuestra una conciencia aguda de las posibles repercusiones a largo plazo y subraya el compromiso de asegurar la estabilidad financiera y el éxito sostenido en el futuro.
	\item Según el organigrama la autoridad responsable en el área de \acrshort{sgodur} (Sub-Gerencia de Obras y Desarrollo Urbano y Rural) es el ing. Ricardo H. Pinto Yupanqui, quien brindó la disponibilidad a atender las dudas y compartió sus conocimientos,  experiencias , consejos y recomendaciones cuando se requirieron.

	\item Existen restricciones, tales como el tope presupuestario que se disponen para solucionar un problema de infraestructura, no siendo suficiente con lo cual no se puede usar el recurso para resolver como debería ser.
\end{enumerate}
\section{Unidad de estudios y proyectos}
\begin{enumerate}
	\item La \acrlong{uep} oficina de la unidad de estudios y proyectos es la encargada de formular los proyectos, fichas de los mantenimientos, \acrshort{ioarr} ,siempre que esté en la capacidad logística y staff de profesionales capaces absolver los requerimientos que pueda surgir como consecuencia de los mismos, en este caso esta oficina dispone de una capacidad muy limitada ya que es una entidad es pequeña en cuanto a recursos.
	\item Los requerimientos para poder trabajar en estudios y proyectos en general son conocimientos en análisis estructural, concreto armado y demás cursos de pre-grado, así mismo también son necesarios el dominio de los diferentes software de ingeniería tales como:ETABS, SAFE, SAP 2000 y Robot Structural, no deja de ser de suma importancia el dominio de software de dibujo y modelado tales como: AutoCAD, Civil 3D, Revit.
	\item Definitivamente, un aspecto destacado que marca la diferencia en esta dinámica laboral son las habilidades blandas, en especial las habilidades interpersonales. Dado que surgen inevitablemente problemas y conflictos, es esencial contar con la capacidad de analizar y coordinar con las diversas partes involucradas, ya sean interesadas u opositoras, sin empeorar la situación. La habilidad para involucrar a todas las partes y hacerlas sentir parte de la solución es clave para gestionar eficazmente los desafíos y mantener un ambiente de trabajo productivo y armonioso.
	\item Es evidente que en esta oficina, cada proyecto requiere una perspectiva multidisciplinaria, ya que se depende de una variedad de especialidades para su ejecución. Esto subraya la importancia de contar con un equipo diversificado de expertos que puedan aportar sus conocimientos y habilidades únicas para abordar los desafíos de manera integral y efectiva. Esta sinergia entre diferentes disciplinas sin duda enriquece la calidad y el éxito de los proyectos que se emprenden en esta oficina.
	\item La lectura y comprensión de las normas internacionales, nacionales y manuales que rigen en nuestro país es una práctica fundamental en el ámbito laboral. Estas fuentes son invaluables al enfrentarse a situaciones ambiguas o cuestionamientos durante el trabajo, ya que proporcionan pautas claras y confiables para tomar decisiones informadas y asegurar el cumplimiento normativo. Por lo tanto, se vuelve imprescindible contar con un conocimiento actualizado de estas regulaciones para garantizar la legalidad y calidad en las acciones emprendidas.
\end{enumerate}

\section{Entorno laboral}
\begin{enumerate}
	\item Eh aqui donde se complican algunas cosas por diferentes causales tales como: Requisitos vagamente definidos, exigencias sin tener en cuenta los costos que implican, cambios a última hora, etc.
	\item En ocasiones sucede que el cliente opina basado en suposiciones y/o emociones, en estos casos es bueno explicarle al cliente de manera profesional.
	\item Sucede muy a menudo que los clientes solicitan modelos BIM a todos los profesionales y se niegan a aceptar trabajos realizados en software convencionales, en estos casos en necesario aclarar mediante directivas bien definidas antes de iniciar los trabajo y las implicancias de estos mismos.
	\item En lo que respecta a los demás aspectos, fue un ambiente con personas amables, colaborativos y cada uno con diferentes virtudes.
\end{enumerate}